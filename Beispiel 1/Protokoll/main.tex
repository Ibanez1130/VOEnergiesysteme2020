\documentclass{eegreport}

%% my usepackages
\usepackage{ltablex}

% external files and paths
\addbibresource{Literature.bib} %% remove, if using BibTeX instead of biblatex
\graphicspath{{graphics/}}

%% general metadata:
\newcommand{\mytitle}{Report Template}  %% also used for PDF metadata (hyperref)
\newcommand{\mysubject}{My Lecture}  %% also used for PDF metadata (hyperref)
\newcommand{\myauthors}{Author Alpha, Author Beta, Author Charlie, Author Gamma}  %% also used for PDF metadata (hyperref)
\newcommand{\mysupervisor}{Mr. Supervisor}  %% also used for PDF metadata (hyperref)
\newcommand{\mydate}{January 1 2020}  %% also used for PDF metadata (hyperref)

\newcommand{\mykeywords}{KEYWORDS}  %% also used for PDF metadata (hyperref)

%% this information is used only for generating the title page:
\newcommand{\myuniversity}{Vienna University of Technology} %% your university/school
\newcommand{\myinstitute}{Institute of Energy Systems and Electrical Drives} %% affiliation
\newcommand{\myworkinggroup}{Energy Economics Group} %% working group
\newcommand{\mytown}{Wien} %% your home town
\newcommand{\mymonth}{October} %% month you are handing in
\newcommand{\myyear}{2015} %% year you are handing in

%% additional information for generic_documentation title page
\allowdisplaybreaks


\begin{document}


\mytitlepage
\pagenumbering{Roman} 
\tableofcontents 
\pagenumbering{arabic} 


\section{Introduction}

Very well prepared tutorial is available \citeurl{TUG.2015}\footcite{TUG.2015}. 

Example citations are \citetitle{APG.11.2013}\footcite{APG.11.2013} from \citeauthor{APG.11.2013} and \citetitle{Joskow.2005}\footcite{Joskow.2005} from \citeauthor{Joskow.2005}. Extendend footnote citation can be made by \footcite[For further information see at][page 17]{Joskow.2005}. In this document citation is managed by Biber, how to adjust you editor can be found at \citeurl{texstackexchange.2015}\footcite{texstackexchange.2015}. For \TeX Studio\footnote{Can be found at \citeurl{Texstudio.2015} \cite{Texstudio.2015}} it works by\footcite{texstackexchange.2015} 
\begin{myquote}
In the current release (2.6 branch), TeXstudio's build process ('Build \& View') by default runs pdfLaTeX but not a bibliography tool, which you need to do separately. There is also a need to change the settings to run Biber rather than BibTeX for creating a bibliography. Thus the steps required are as follows:
\begin{enumerate}
\item In the TeXstudio preferences ('Preferences ...' on the Mac or 'Options $ \rightarrow $ Configure TeXstudio' on Windows)\footnote{For versions up from 2.8 in chose the tab 'Optionen' $ \rightarrow $ 'Erzeugen'. }, choose the Build tab and alter the 'Default Bibliography' to 'Biber'. Save and close the preferences.
\item Run 'Build \& View' from the 'Tools' menu (or press the two green arrows icon), which will create a PDF but with the bibliography not completed
\item Run 'Bibliography' from the 'Tools' menu.
\item Run 'Build \& View' again: the bibliography will appear in the PDF.
\end{enumerate}
It is possible to set up TeXstudio in alternative ways to achieve the same effect. The key is that you have to ensure that the is a sequence
\begin{enumerate}
\item \LaTeX
\item Biber
\item \LaTeX
\end{enumerate}
which can be done 'by hand' (as I have) or can be automated in various ways. Note that the same general idea applies whatever editor is used: this is a feature of LaTeX and not of the editor.
\end{myquote}

\myfig{APG_grid_2012.pdf}%% filename in figures folder
       {width=0.75\textwidth}%% maximum width/height, aspect ratio will be kep
       {APG power grid and border transmission lines based on \germanwordQM{\citetitle{APG.11.2013}}, \cite{APG.11.2013}}%% caption
       {}%% optional (short) caption for list of figures
       {APG_grid_2012}%% label
       {htb}%% figure position
       
\myfigpdftex{market_participants}%% filename in figures folder
       {0.9}%% maximum width/height, aspect ratio will be kept
       {Communication and data flows between market participants.}%% caption
       {}%% optional (short) caption for list of figures
       {market_participants}%% label
       {htb}%% figure position



Figures can be inserted as PDF (see Figure \ref{fig:APG_grid_2012}) or drawn/saved via Inkscape\footnote{For further information/download see at \citeurl{Inkscape.2015} \cite{Inkscape.2015}2} (see Figure \ref{fig:market_participants}). To embed an Inkscape drawing inside a \LaTeX document draw it and save it as PDF. After a window pops up you have to confirm the entry 'PDF + LATEX: ...' and save the PDF. If you look in the figure's folder you will find a '.pdf\_tex' and '.pdf' file. Now you just have to embed the '.pdf\_tex' file in you \LaTeX code/file. 

If mathematical formulas, equations, etc. are used in the work, consider following points:
\begin{itemize}
	\item Formulas are components of sentences and must be a part of these. For example: According to Einstein (1905), the 'rest energy' of a physical system with mass $ m $ 
	\begin{equation}\label{equ:Einstein}
	E_0 = m c^2,
	\end{equation}
	is calculated with the speed of light $ c = \SI{299792458}{m/s} $.
	\item For ease of readability formulas are written in a separate paragraph and numbered consecutively. By numbering, it is possible to refer to formulas (eg. see equation \ref{equ:Einstein}).
	\item Variables are italicized, while units are to be written in plain text form (eg. $ c = \SI{299792458}{m/s} $).
	\item Linear Algebra: vectors are shown in the form (lowercase and boldface)
		\begin{equation}
			\vect{x} = \begin{bmatrix} x_1, x_2, \ldots x_N \end{bmatrix}^T
		\end{equation}
		and matrices in the form of (capital letters and bold font)
		\begin{equation}
			\mat{X} = \begin{bmatrix} x_1    & 0      & \dots & 0 \\
									    0    & x_2    & \dots & 0 \\ 
									  \vdots & \vdots & \ddots&   \\      
									    0    & 0      &       & x_N\end{bmatrix}.
		\end{equation}
	\item An optimization problem's standard form can be written as
		\begin{subequations}
		\begin{align}
		\underset{\vect{x}}{\text{min}} & \quad f(\vect{x}) \\
		\text{s.t.} & \quad g_i(\vect{x}) = b_i, \; i = 1, \ldots, p\\
					    & \quad h_i(\vect{x}) \leq 0, \; i = 1, \ldots, q.
		\end{align}
		\end{subequations}
\end{itemize}


\section{My Section 1}
\subsection{My Subsection 1}

\subsection{My Subsection 2}

\subsection{My Subsection 3}

\section{My Section 2}

\end{document}