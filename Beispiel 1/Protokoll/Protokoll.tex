\documentclass[a4paper,12pt]{article}
\usepackage[utf8]{inputenc}
\usepackage[ngerman]{babel}
\usepackage[a4paper, left=2.5cm, right=2.5cm]{geometry}
\usepackage{graphicx}
\usepackage{subcaption}
\usepackage{fancyhdr}
\usepackage{pdfpages}
\usepackage{listings}
\usepackage{float}

\pagestyle{fancy}
\lstset{
	language=Matlab,
	breaklines=true,
	morekeywords={matlab2tikz},
	keywordstyle=\color{blue},
	morekeywords=[2]{1}, keywordstyle=[2]{\color{black}},
	identifierstyle=\color{black},
	stringstyle=\color{mylilas},
	commentstyle=\color{mygreen},
	showstringspaces=false,
	mathescape=true
	emph=[1]{for,end,break},emphstyle=[1]\color{red},
}

\lhead{Sonneneinstrahlung und Photovoltaik Teil 1}
\chead{}
\rhead{Gruppe D}

\begin{document}
	\includepdf{Protokoll_titlepage.pdf}
	
	\newpage
	%Inhaltsverzeichnis
	\tableofcontents
	
	\newpage
	\section{Aufgabenstellung}
	\subsection{Aufgabe 1.1}
	Aufgabe 1.1 befasst sich mit einer PV-Anlage mit folgenden Parametern:
	\begin{itemize}
		\item Der Standort ist Wien ($48.2^{\circ}$N, $16.3^{\circ}$O).
		\item Die installierte Leistung ist $1kWp$.
		\item Der Neigungswinkel der PV-Anlage beträgt $20^{\circ}$.
		\item Der Azimut der Anlage ist $180^{\circ}$ Süden.
		\item Der Modulwirkungsgrad $\eta_{Modul}$ ist $0.17$.
		\item Sonstige Verluste $\eta_{sonst}$ (Reflexion, Temperatur, Wechselrichter, etc.) werden mit dem Wert $0.8$ eingerechnet.
		\item Die Strahlungsdaten für den Standort sind in der Datei $Strahlung.mat$ gegeben.
		\item Die Zeit in Viertelstunden-Werten ist in der Datei $time.mat$ gegeben.
		\item Die Errechnung des Sonnenstandes erfolgt mit der in der Datei $SonnenstandTST.m$ zur Verfügung gestellten Funktion $SonnenstandTST()$.
	\end{itemize}
	Zusätzlich werden folgende Annahmen getroffen:
	\begin{itemize}
		\item Standardtestbedingungen zur Bestimmung des Modulwirkungsgrades bzw. der Nennleistung $P_{peak}$ (in W) bei $25^{\circ}$ Modultemperatur.
		\begin{equation}
			P_{peak}=R_{STC}*A*\eta_{Modul}
		\end{equation}
		\item Vereinfachte Annahme für die Bestimmung des Ertrags der Anlage im Modell.
		\begin{equation}
			E_{ges}=G_{geneigt}*A*\eta_{Modul}*\eta_{sonst}
		\end{equation}
		\item Konstanter Wirkungsgrad.
		\item Erträge bei einem Höhenwinkel unter $5^{\circ}$ werden vernachlässigt.
		\item Konstante Einstrahlung in den 15 Minuten Intervallen.
		\item Norden $5^{\circ}$, Osten $90^{\circ}$, Süden $180^{\circ}$, Westen $270^{\circ}$.
	\end{itemize}
	Die Aufgaben lauten:
	\begin{itemize}
		\item[a)] Erstellen Sie ein Modell, das für den gegebenen Sonnenstand und die Einstrahlungswerte (Diffus- und Direktstrahlung) auf eine horizontale Fläche den Ertrag der PV-Anlage nach Angabe der installierten Leistung in $kW_{peak}$ und der Ausrichtung der Anlage (Azimut und Neigungswinkel) modelliert. Verwenden Sie dazu das isotrope Einstrahlungsmodell.
		\item[b)] Berechnen Sie mit Hilfe der Funktion aus a) den gesamten Jahresertrag 2005 und die Volllaststunden einer $1kWp$ Anlage in Wien.
	\end{itemize}
	\subsection{Aufgabe 1.2}
	Die Unterpunkte der Aufgabe 1.2 lauten:
	\begin{itemize}
		\item[a)] Erstellen Sie die Leistungsdauerlinie der PV-Erzeugung über das Jahr. Sortieren Sie dazu die erzeugte Leistung vom Maximum bis zum Minimum.
		\item[b)] Plotten Sie die monatlichen Erträge der PV-Erzeugung (12 Werte).
		\item[c)] Ermitteln Sie jeweils die 5 Tage mit der minimalen und der maximalen PV-Erzeugung. Geben Sie die Tage (Datum) und den energetischen Ertrag dieser Tage an.
		\item[d)] Stellen Sie in einem Diagramm die Anteile der Diffus-, Direkt- und der reflektierten Strahlung an jedem der 365 Tage dar (verwenden Sie dazu das File $plotStrahlungsanteile.m$).
		\item[e)] Berechnen Sie die durchschnittliche Stromproduktion für jede Stunde am Tag für die Monate Juni und Dezember. Erstellen Sie ein Diagramm mit Boxplots der Erzeugung für jede Stunde des Tages für die jeweiligen Monate.
		\begin{itemize}
			\item Jeder Stundenwert besteht aus der Summe von vier Viertelstundenwerten.
			\item Jeder Monat wird durch eine Matrix mit den Abmessungen $Stunden x Tage$ dargestellt.
			\item Der Input eines Boxplots ist eine Matrix.
		\end{itemize}
	\end{itemize}
	\newpage
	\section{Berechnungen - Aufgabe 1.1}
	\subsection{Beschreibung der Winkel}
	\begin{figure}[H]
		\centering
		\includegraphics[width=12cm]{img/Winkel}
		\caption{Darstellung des Einfall- und Moduleinfallswinkel.}
	\end{figure}
	\begin{itemize}
		\item $\alpha_S$ - \textbf{Sonnenazimut}. Der Sonnenazimut ist der Winkel zwischen der geographischen Nordrichtung und dem Vertikalkreis durch den Sonnenmittelpunkt. Er ist abhängig von der geographischen Breite des Standorts, der Jahreszeit und der Tageszeit.
		\item $\gamma_S$ - \textbf{Sonnenhöhe}. Die Sonnenhöhe ist der Winkel zwischen dem Sonnenmittelpunkt und der Horizontalebene vom Beobachter. Er ist ebenfalls abhängig von der geographischen Breite des Standorts, der Jahreszeit und der Tageszeit. 
		\item $\alpha_E$ - \textbf{Modulazimut}. Der Modulazimut ist der Winkel der die Modulausrichtung gegenüber dem geographischen Nordpol angibt.
		\item $\gamma_E$ - \textbf{Modulneigungswinkel.}
		\item $\Theta_{gen}$ - \textbf{Moduleinfallswinkel geneigt}. Der Einfallswinkel der Sonnenstrahlung auf eine geneigte Fläche.
		\item $\Theta_{hor}$ - \textbf{Moduleinfallswinkel horizontal}. Der Einfallswinkel der Sonnenstrahlung in horizontaler Richtung.
		\item $Zenit$ - \textbf{Zenit}. Der Zenit steht normal auf den "Horizont".
	\end{itemize}
	\subsection{Berechnung des Moduleinfallswinkels $\Theta_{gen}$}
	Der Moduleinfallswinkel $\Theta_{gen}$ ist für die Berechnung der einzelnen Strahlungsanteile relevant. Er ist von der Südausrichtung des Moduls abhängig.\\
	In unserem Fall beträgt die Südausrichtung $180^{\circ}$. Daraus folgt die Formel zur Berechnung des Moduleinfallswinkels zu
	\begin{equation}
		\Theta_{gen} = \arccos{[-\cos{(\gamma_S)}*\sin{(\gamma_E)}*\cos{(\alpha_S-\alpha_E-180^{\circ})}+\sin{(\gamma_S)}*\cos{(\gamma_E)}]}
	\end{equation}
	Der MATLAB Code, zur Berechnung des Moduleinfallswinkels $Theta_{gen}$ auf eine geneigte Fläche lautet wie folgt:
	\begin{lstlisting}
		acosd($-$cosd(sHoehenwinkel).*sind(pvHoehenwinkel).*cosd(sAzimut $-$ pvAzimut $-$ 180)+sind(sHoehenwinkel).*cosd(pvHoehenwinkel));
	\end{lstlisting}
	In dieser Formel ist hervor zu heben, dass zur Berechnung jeweils die Funktionen sind, cosd und acosd genutzt wurden. Die Besonderheit liegt darin, dass diese Funktionen den Übergabeparameter in Grad erwarten, wohingegen sin, cos und acos den Winkel in Radiant erwarten.\newline
	In der Funktion werden folgende Variablen genutzt:
	\begin{itemize}
		\item \textbf{sAzimut} - Der Sonnenazimutalwinkel der Sonne. Dieser Wert wird mit Hilfe der Funktion SonnenstandTST berechnet.
		\item \textbf{sHoehenwinkel} - Der Höhenwinkel der Sonne. Dieser Wert wird mit Hilfe der Funktion SonnenstandTST berechnet.
		\item \textbf{pvAzimut} - Der Azimutalwinkel der PV-Anlage. In unserem Fall entspricht $\alpha_E$ einem Winkel von $270^{\circ}$.
		\item \textbf{pvHoehenwinkel} - Der Höhenwinkel der PV-Anlage. Dieser entspricht dem Modulneigungswinkel. In unserem Fall entspricht $\gamma_E$ einem Winkel von $20^{\circ}$.
	\end{itemize}
	\subsection{Berechnung der Strahlungsanteile auf eine geneigte Fläche}
	Im Falle einer geneigten PV-Anlage sind drei Strahlungsanteile relevant: direkte Strahlung, diffuse Strahlung und reflektierte Strahlung. (Im Falle einer horizontalen PV-Anlage würden sich die Strahlungsanteile auf die direkte und die diffuse Strahlung begrenzen, da die reflektierte Strahlung über $\frac{1-\cos{(\gamma_E)}}{2}$ vom Modulneigungswinkel abhängig ist. Im Falle eines Modulneigungswinkels ergibt sich somit für die reflektierte Strahlung ein Wert von $0$.)
	\subsubsection{Direkte Strahlung}
	Die direkte Strahlung ist der Anteil, der direkt auf der PV-Anlage auftrifft.\newline
	Die Berechnung des direkten Strahlungsanteils erfolgt über die Formel
	\begin{equation}
		E_{dir,gen}=E_{dir,hor}*{max(0,\frac{\cos{\Theta_{gen}}}{\sin{\gamma_S}})}.
	\end{equation}
	$E_{dir,hor}$ entspricht dabei der gemessenen Direktstrahlung auf eine horizontale Fläche. Die Daten für $E_{dir,hor}$ sind in der Datei $Strahlung.mat$ gegeben.
	Der MATLAB Code für die Berechnung lautet
	\begin{lstlisting}
	DirectGen = Strahlung.DirectHoriz.*max(0, (cosd(pvModuleinfallswinkel)./sind(sHoehenwinkel)));
	\end{lstlisting}
	\subsubsection{Diffuse Strahlung}
	Der diffuse Strahlungsanteil entspricht der Strahlung, die am Weg durch die Erdatmosphäre mit ihr wechselwirkt (z.B. mit Wolken).
	Die diffuse Strahlung kann als eine Hohlhalbkugel betrachtet werden, die sich über der PV-Anlage befindet und Licht emittiert. (Isotropes Diffusstrahlungsmodell)\newline
	Die Berechnung des diffusen Strahlungsanteils erfolgt über die Formel
	\begin{equation}
		E_{diff,gen,iso}=E_{diff,hor}*\frac{1+\cos{(\gamma_E)}}{2}.
	\end{equation}
	Daraus ergibt sich folgender MATLAB Code:
	\begin{lstlisting}
	DiffusGen = Strahlung.DiffusHoriz.*(1+cosd(pvHoehenwinkel))./2;
	\end{lstlisting}
	\subsubsection{Reflektierte Strahlung}
	Bei der reflektierten Strahlung handelt es sich um den Strahlungsanteil, der zuerst vom Boden reflektiert wird und dann auf dem PV Modul auftrifft.\newline
	Daraus folgt die Formel zur Berechnung des reflektierten Strahlungsanteils zu
	\begin{equation}
		E_{refl,gen}=E_{G,hor}*A*\frac{1-\cos{(\gamma_E)}}{2}.
	\end{equation}
	$A$ entspricht in dieser Formel dem Albedo-Wert. Dieser entspricht dem Verhältnis der auf eine Fläche einfallenden Strahlung und der von der Fläche reflektierten Strahlung. Ist dieser Wert für eine Fläche nicht bekannt, kann er mit dem Wert $0.2$ angenommen werden.\newline
	Der daraus resultierende MATLAB Code, zur Errechnung der reflektierten Strahlung, lautet
	\begin{lstlisting}
	ReflectedGen = Strahlung.Reflected.*0.2.*(1$-$cosd(pvHoehenwinkel))./2;
	\end{lstlisting}
	\subsection{Berechnung der gesamten Strahlung auf eine geneigte Fläche}
	Wie bereits eingangs erwähnt, setzt sich die gesamte Strahlung auf eine geneigte Fläche aus dem direkten, dem diffusen und dem reflektierten Strahlungsanteil zusammen.\newline
	Daraus ergibt sich
	\begin{equation}
		E_{G,gen}=E_{dir,gen}+E_{diff,gen,iso}+E_{refl,gen}.
	\end{equation}
	Unter Berücksichtigung des obigen MATLAB Codes ergibt sich für die Berechnung der gesamten Strahlung auf eine geneigte Fläche
	\begin{lstlisting}
	GesGen = DirectGen + ReflectedGen + DiffusGen;
	\end{lstlisting}
	Als zusätzliche Annahme wurde definiert, dass Erträge bei einem Höhenwinkel unter $5^{\circ}$ nicht berücksichtigt werden sollen.\newline
	Diese Annahme lässt sich durch folgende Formel umsetzen:
	\begin{lstlisting}
	GesGen(sHoehenwinkel < 5) = 0;
	\end{lstlisting}
	\subsection{Berechnung des Ertrags}
	Der gesamte Ertrag $E_{ges}$ der Anlage errechnet sich durch
	\begin{equation}
		E=E_{G,gen}*A*\eta_{Modul}*\eta_{sonst}.
	\end{equation}
	\begin{itemize}
		\item \textbf{$E_{G,gen}$} entspricht der gesamten Einstrahlung (dem gesamten Ertrag) auf die geneigte PV-Anlage.
		\item \textbf{$A$} entspricht der Fläche der PV-Anlage.
		\item \textbf{$\eta_{Modul}$} ist der Modulwirkungsgrad (in unserem Fall $0.17$).
		\item \textbf{$\eta_{sonst}$} sind sonstige Verluste, die durch Reflexion, die Temperatur, Wechselrichter, etc. auftreten. (In unserem Fall $0.8$).
	\end{itemize}
	Da die in der Variable $E_{G,gen}$ errechneten Werte in 15 Minuten Intervalle aufgeteilt sind, müssen wir in MATLAB eine zusätzliche Korrektur (ein Multiplikator mit dem Wert $0.25$) in die Formel einfügen:
	\begin{lstlisting}
		Eges = GesGen.*0.25.*pvFlaeche.*pvWirkungsgrad.*pvVerluste;
	\end{lstlisting}
	\newpage
	\section{Ergebnisse - Aufgabe 1.1}
	\section{Berechnungen - Aufgabe 1.2}
	\section{Ergebnisse - Aufgabe 1.2}
\end{document}